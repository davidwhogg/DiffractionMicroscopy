%% This file is part of the DiffractionMicroscopy project
%% Copyright 2016 David W. Hogg

% To-Do list
% ----------
% - Write.

% Style notes
% -----------
% - Always \eqnarray never \equation.

\documentclass[12pt]{article}
\usepackage{amsmath, amssymb, mathrsfs, hyperref, graphicx}
\input{vc}

% useless formatting
\linespread{1.08333} % 10/13 spacing
\setlength{\parindent}{1.08333\baselineskip}
\setlength{\parskip}{0ex}
\newcommand{\hoggcaption}[1]{\caption{\textsf{#1}}}
\renewcommand{\thefigure}{\textsf{\arabic{figure}}}
\renewcommand{\figurename}{\textsf{Figure}}
\newcommand{\equationname}{equation}

% math definitions
\newcommand{\dd}{\mathrm{d}}

\begin{document}\sloppy\sloppypar

\section*{\raggedright%
  Derivatives of the likelihood function (with respect to parameters)
  for diffraction microscopy with noisy images at unknown orientations.}
\noindent
David W. Hogg%
\footnote{Simons Center for Data Analysis, Simons Foundation}%
\footnote{Center for Cosmology and Particle Physics, Department of Physics, New York University}%
\footnote{Center for Data Science, New York University}%
\footnote{david.hogg@nyu.edu}

\bigskip

\paragraph{Abstract:}
I write down a likelihood for diffraction microscopy imaging data,
where each image is of the same molecule, but each image contains a
small number of photons, and each image is of the molecule at an
unknown, random orientation.
In this work, at fixed orientation, the likelihood function is
parameterized by linear parameters (densities in three-dimensional
boxels in real space, or Fourier coefficients, or any other linear
representation), but the likelihood function becomes non-linear in
those parameters when the orientation is marginalized out.
Marginalization is performed by a sum over a (random or uniform)
sampling in orientation space.

\section{Introduction}

In diffraction imaging, the fundamental observable is the squared norm
of a slice of the Fourier transform of the object of interest.
In the noisy regime---but when the orientation of the sample in the
beam is known---the detected photons are independent draws from a
variable-rate Poisson process, with rate set by the squared norm of
the relevant slice of the Fourier transform (probably censored by a
beam stop).
When the orientation of the sample in the beam is unknown, the group
of detected photons can no longer be thought of as being independent
draws from a simple process.
While being conditionally independent given the orientation, they are
dependent when the orientation is thought of as an unobserved latent
variable.
This emerges because the orientation controls the slice, and the slice
controls the rate function for the Poisson process; the various
photons in a single image or exposure are generated by the same slice,
but the plane of the slice is not known.

For the purposes of definiteness, we will think of the real-space
object as being described by a density $\rho(x)$, where $x$ is
a three-vector position in real space (units of length).
This density has a Fourier transform $\hat{\rho}(k)$, where the hat
indicates the Fourier transform, and $k$ is a three-vector wave number
(units of inverse length).

We will make use of various properties of the density $\rho(x)$ and of
its Fourier transform $\hat{\rho}(k)$:
One might be that the density $\rho(x)$ has compact support---is
non-zero in a finite domain---which makes the Fourier transform
$\hat{\rho}(k)$ smooth in some corresponding way.
Another is that the density $\rho(x)$ will be real, which makes
$\hat{\rho}(k)$ symmetric in the sense of
$\hat{\rho}(k)=\hat{\rho}(-k)$.
Another might be that the density $\rho(x)$ is non-negative everywhere
(this will often---but not always---be true).
Non-negativity in real space has non-trivial implications for Fourier
space.

\section{Likelihood function}

For us, a probabilistic \emph{model} is a likelihood function (a
probability for the data given parameters) and a prior pdf over (at
least) any nuisance parameters.
This requires that we obtain a parameterization for (either) the
density $\rho(x)$ (or its Fourier transform or some function of
either).
Because our objective is to obtain a description of the real-space
function $\rho(x)$, we will parameterize this function simply as a
linear mixture of $M$ components
\begin{eqnarray}
       \rho(x)  &=& \sum_{m=1}^M a_m\,     g_m(x)
  \\
  \hat{\rho}(k) &=& \sum_{m=1}^M a_m\,\hat{g}_m(k)
\quad ,
\end{eqnarray}
where the $a_m$ are coefficients, the $g_m(x)$ are real-space basis
functions that linearly sum to make the density, and the
$\hat{g}_m(k)$ are the Fourier transforms of those basis functions.

Need a terminology for projecting and slicing.

foo

\section{Derivatives}

bar

\end{document}
