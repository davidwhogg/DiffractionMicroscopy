%% This file is part of the DiffractionMicroscopy project
%% Copyright 2016 David W. Hogg

% To-Do list
% ----------

% Style notes
% -----------

\documentclass[12pt]{article}
\usepackage{amsmath, amssymb, mathrsfs, hyperref, graphicx}
\input{vc}

% useless formatting
\linespread{1.08333} % 10/13 spacing
\setlength{\parindent}{2\baselineskip}\addtolength{\parindent}{-1.25ex}
\setlength{\parskip}{0ex}
\newcommand{\hoggcaption}[1]{\caption{\textsf{#1}}}
\renewcommand{\thefigure}{\textsf{\arabic{figure}}}
\renewcommand{\figurename}{\textsf{Figure}}
\newcommand{\equationname}{equation}

% text definitions
\newcommand{\doi}[1]{{\footnotesize \url{http://doi.org/#1}}}

% math definitions
\newcommand{\normal}{\mathscr{N}}
\newcommand{\Poisson}{\mathscr{P}}
\newcommand{\sqnorm}[1]{|{#1}|^2}
\newcommand{\unitvec}[1]{\hat{#1}}
\newcommand{\ehat}{\unitvec{e}}
\newcommand{\xhat}{\unitvec{x}}
\newcommand{\yhat}{\unitvec{y}}
\newcommand{\zhat}{\unitvec{z}}
\newcommand{\transpose}{^{\mathsf{T}}}
\newcommand{\given}{\,|\,}
\newcommand{\setof}[1]{\left\{{#1}\right\}}
\newcommand{\dd}{\mathrm{d}}
\DeclareMathOperator{\trace}{tr}

\begin{document}\sloppy\sloppypar

\section*{\raggedright%
  Derivatives of the likelihood function (with respect to parameters)
  for diffraction microscopy with noisy images at unknown orientations.}
\noindent
David W. Hogg%
\footnote{Simons Center for Data Analysis, Simons Foundation}%
\footnote{Center for Cosmology and Particle Physics, Department of Physics, New York University}%
\footnote{Center for Data Science, New York University}%
\footnote{david.hogg@nyu.edu}

\bigskip

\paragraph{Abstract:}
I write down a likelihood for diffraction microscopy imaging data,
where each image is of the same molecule, but each image contains a
small number of photons, and each image is of the molecule at an
unknown, random orientation.
In this work, at fixed orientation, the likelihood function is
parameterized by linear parameters (densities in three-dimensional
boxels in real space, or Fourier coefficients, or any other linear
representation), but the likelihood function becomes non-linear in
those parameters when the orientation is marginalized out.
Marginalization is performed by a sum over a (random or uniform)
sampling in orientation space.

\section{introduction}

In diffraction imaging, the fundamental observable is the squared norm
of the Fourier transform of the object of interest.
In the noisy regime, the detected photons are independent draws from a
variable-rate Poisson process, with rate set by the squared norm of
the Fourier transform (probably censored by a beam stop).
When the angles are unknown, the detected photons are drawn from a
mixture of such Poisson processes, with the mixture set by the
integral over all possible orientations (with some prior).

\section{likelihood function}

foo

\section{derivatives}

bar

\end{document}
