%% This file is part of the PhaseRetrieval project.
%% Copyright 2015 David W. Hogg (NYU)

%% To-do:
%% ------
%% - Address all ``DWH'' instances.

%% Style notes:
%% ------------
%% - Use \pars and \angles not \Theta and \phi and so on.

\documentclass[12pt, letterpaper]{article}

\newcommand{\foreign}[1]{\textsl{#1}}
\newcommand{\pars}{\Theta}
\newcommand{\image}{y}
\newcommand{\angles}{\phi}

\begin{document}

\section*{Reconstruction of the Fourier Transform norm \\
  from noisy slices taken at unknown angles}
\smallskip
\noindent
David W. Hogg (NYU)
\bigskip

\paragraph{Abstract:}
In certain kinds of diffraction-microscopy problems, the sample being
imaged is only in the device for a short time, and at unknown
orientation (Euler angles), leading to a noisy image of one 2-d slice
of the squared norm of the full 3-d Fourier Transform.
In principle, many of these noisy slice images can be combined to
reconstruct the full 3-d function.
However, the fact that the orientations are unknown makes this
difficult, and perhaps impossible at some individual-image
signal-to-noise ratios.
Here we present a fully probabilistic approach to this problem, which
reconstructs the full 3-d function, marginalizing out the Euler
angles.
We find that reconstruction of XXX quality can only be obtained with
YYY data and ZZZ.

\section{Introducion}

DWH: This problem, apparently, is an important problem.  State it clearly
and cite some literature (see references below).

DWH: We have to decide the scope: Are we just getting the squared norm of
the FT, or are we going to get the phases too?

\section{Method}

The assumptions of our method are as follows:
\begin{enumerate}
\item We have $N$ images $\image_n$, each of which contains a square
  pixel grid of counts of photons; that is, each image is a
  two-dimensional (2-d) photon histogram with equal-sized bins.
\item Photons are Poisson distributed, in the sense that the count in
  each pixel $\image_{nm}$ of image $\image_n$ is a Poisson draw with
  a mean equal to the true 2-d function for image $\image_n$,
  evaluated at [DWH: or integrated over?] pixel $m$.
\item The true 2-d function for image $\image_n$ is a rotated slice of
  the true three-dimensional (3-d) function, sliced and rotated at
  Euler angles $\angles_n$ (each $\angles_n$ contains all three
  angles), and projected [BY WHAT] onto the detector plane.  The
  angles $\angles_n$ are not known \foreign{a priori} and they are
  nuisance parameters for the reconstruction.
\item The true 3-d function is the squared norm of the Fourier
  Transform of some kind of density in 3-d real space.  The density
  has finite support in real space and the autocorrelation of the
  density---or equivalently the squared norm of its Fourier
  Transform---can be represented by some compact set of parameters
  $\pars$.
\end{enumerate}

\section{Experiments}

\section{Discussion}

\paragraph{Acknowledgments:}
It is a pleasure to thank...
Grant numbers...

\section*{References}\raggedright
\begin{trivlist}
\item
Bortel, G., Faigel, G., Tegze, M., 2009,
Classification and averaging of random orientation single macromolecular diffraction patterns at atomic resolution,
J Struct Biol 166(2) 226--233.
\item
Huldt, G., Szoke, A., Hajdu, J., 2003,
Diffraction imaging of single particles and biomolecules,
J Struct Biol 144(1--2) 219--227.
\item
Martin, A. V., 2014,
The correlation of single-particle diffraction patterns as a continuous function of particle orientation,
Philos Trans R Soc Lond B Biol Sci 369(1647) 20130329 doi:10.1098/rstb.2013.0329.
\item
Tegze, M., Bortel, G., 2012,
Atomic structure of a single large biomolecule from diffraction patterns of random orientations,
J Struct Biol 179(1) 41--45 doi: 10.1016/j.jsb.2012.04.014.
\end{trivlist}

\end{document}
